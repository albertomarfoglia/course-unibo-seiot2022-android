\documentclass{beamer}

\input{style/ancro-slide-format}

\title[Android -- 3A -- HTTP, Bluetooth]{Android Programming}
\subtitle{HTTP and Bluetooth communication}
\date[ver. 1.0 (20220505)]{Embedded Systems and Internet of Things\\A.A. 2021 -- 2022}

\begin{document}

  \begin{frame}
    \titlepage
  \end{frame}

  \newcommand\blfootnote[1]{%
  \begingroup
  \renewcommand\thefootnote{}\footnote{#1}%
  \addtocounter{footnote}{-1}%
  \endgroup
}

\begin{frame}{Thanks}
    \centering
    \begin{itemize}
      \item \large{Professor Angelo Croatti}
    \end{itemize}

    \blfootnote{\url{https://www.unibo.it/sitoweb/a.croatti/en}}
    \blfootnote{}
\end{frame}

  \section{HTTP Communication}

  \subsection{HTTP Requests}

  \begin{frame}[fragile]{HTTP requests}
    \begin{block}{Overview}
      \begin{itemize}
        \item The Java HTTP client (\codeB{java.net.HttpURLConnection})
        facilitates the HTTP requests creation (GET, POST, PUT, DELETE).
        \item Alternatively, Android suggests using the \tcc{Volley} library. 
        \begin{itemize}
          \item \url{https://developer.android.com/training/volley}
        \end{itemize}
      \end{itemize}
    \end{block}

    \begin{exampleblock}{Permissions}
      \begin{lstlisting}[language=XML]
  <uses-permission android:name="android.permission.INTERNET"/>
  <uses-permission
    android:name="android.permission.ACCESS_NETWORK_STATE"/>
      \end{lstlisting}
    \end{exampleblock}

    {\small \textbf{Note} -- Since Android 9.0 (API Level 28), to execute not
    encrypted requests via network, you must specify the
    \codeB{android:usesCleartextTraffic="true"} attribute in the
    \codeB{<application>} field of the manifest file.}  

  \end{frame}

  \begin{frame}[fragile]{Examples}
    \begin{exampleblock}{GET Request}
      \begin{lstlisting}[language=Java]
String doGet(URL url) {
  HttpURLConnection c = (HttpURLConnection) url.openConnection();
  c.setRequestMethod("GET");
  if(c.getResponseCode() == HttpURLConnection.HTTP_OK)
    return readStream(c.getInputStream());
}
      \end{lstlisting}
    \end{exampleblock}

    \begin{exampleblock}{POST Request}
      \begin{lstlisting}[language=Java]
String doPost(URL url, byte[] payload) {
  HttpURLConnection c = (HttpURLConnection) url.openConnection();
  c.setRequestMethod("POST");
  c.setDoOutput(true);
  c.getOutputStream().write(payload);
  if(c.getResponseCode() == HttpURLConnection.HTTP_OK)
    return readStream(c.getInputStream());
}
      \end{lstlisting}
    \end{exampleblock}
  \end{frame}

  \begin{frame}{HTTP requests -- notes}
    \begin{block}{Some notes}
      \begin{itemize}\itemsep20pt
        \item You cannot make HTTP requests on the Main Thread 
        \begin{itemize}
          \item They must be delegated to a Worker thread (ex. via AsyncTask)
        \end{itemize}
        \item \textbf{HTTP requests are intrinsically asynchronous} 
      \begin{itemize}
        \item The response, if any, can arrive after a certain amount of time
        \item During this time, a timeout may be triggered which signals the
        unavailability of the server or network.
      \end{itemize}
      \end{itemize}
    \end{block}
  \end{frame}

\subsection{Verify Connectivity}

  \begin{frame}[fragile]{Verify connectivity}
    \begin{block}{Connectivity manager}
      \begin{itemize}\itemsep20pt
        \item The \codeB{ConnectivityManager} service allows you to check if the
        device is connected to the Internet or not.
      \end{itemize}
    \end{block}
    \begin{exampleblock}{Example}
      \begin{lstlisting}[language=Java]
ConnectivityManager cm = (ConnectivityManager) getSystemService(Context.CONNECTIVITY_SERVICE);
NetworkInfo activeNetwork = cm.getActiveNetworkInfo();

if(activeNetwork.isConnectedOrConnecting()) {
  /* Network is reachable... */
}
      \end{lstlisting}
    \end{exampleblock}
  \end{frame}


\section{Bluetooth Communication}

  \begin{frame}{Bluetooth in Android}
    \begin{block}{Overview}
      \begin{itemize}\itemsep10pt
        \item Android, already from the first versions of the system, includes
        support for the communication stack based on the Bluetooth standard.
        \begin{itemize}
          \item Through the Android Bluetooth API
          \item Both point-to-point and multipoint BT connections are enabled. 
        \end{itemize}
        \item In particular, in Android it's possible to:
        \begin{itemize}
          \item Analyze the radio frequencies to identify other nearby devices
          (\tcc{discovery})
          \item Connect to previously paired devices (\tcc{pairing})
          \item Transfer data from one device to another via the BT stack
          \item Manage multiple connections
        \end{itemize}
      \end{itemize}
    \end{block}
  \end{frame}

\subsection{API and Permissions}

  \begin{frame}{Android Bluetooth API}
    \begin{block}{API summary}
      \begin{itemize}\itemsep10pt
        \item All the Android Bluetooth API are provided through the
        \codeB{android.bluetooth.*} package.
        \item Among others, the most significant are:
        \begin{itemize}\itemsep10pt
          \item \tcc{BluetoothAdapter}: represents the entry-point for every
          interaction based on bluetooth.
          \item \tcc{BluetoothDevice}: represents a remote bluetooth device with
          which an appropriate communication channel has been established.
          \item \tcc{BluetoothSocket} and \tcc{BluetoothServerSocket}: defines the
          bluetooth interfaces to activate TCP-like communication channels. 
        \end{itemize}
      \end{itemize}
    \end{block}
  \end{frame}

  \begin{frame}[fragile]{Permissions}
    \begin{block}{Some notes}
      \begin{itemize}
        \item To use the Bluetooth support offered by the operating system, it is
        necessary to request the relative permissions (\codeB{BLUETOOTH} and
        \codeB{BLUETOOTH\_ADMIN}), declaring them in the File Manifest.
        \item From Android 6.0 onwards it necessary to request an additional permission 
        (\codeB{ACCESS\_FINE\_LOCATION})
        \begin{itemize}
          \item This is a \bothquote{dangerous} permission (user confirmation
          required). 
        \end{itemize}
      \end{itemize}
    \end{block}

  \begin{exampleblock}{BT permissions}
    \begin{lstlisting}[language=XML]
<uses-permission android:name="android.permission.BLUETOOTH"/>
<uses-permission
  android:name="android.permission.BLUETOOTH_ADMIN"/>
<!-- Necessary for device with Android 6.0 or higher --> 
<uses-permission
  android:name="android.permission.ACCESS_FINE_LOCATION"/>
    \end{lstlisting}
  \end{exampleblock}


\end{frame}

\subsection{Initialization}

  \begin{frame}[fragile,allowframebreaks]{Bluetooth Initialization}
    \begin{block}{BluetoothAdaper}
      \begin{itemize}
        \item Through the instance of the default BluetoothAdapter provided by the
        system it is possible: 
        \begin{enumerate}
          \item Check if Bluetooth stack based communication is available for the
          specific device
          \item Request the user to enable it if bluetooth support is available
          but temporarily disabled.
        \end{enumerate}
      \end{itemize}
    \end{block}
    \begin{exampleblock}{1. Check Bluetooth availability}
      \begin{lstlisting}[language=Java]
BluetoothAdapter btAdapter = BluetoothAdapter.getDefaultAdapter();
if(btAdapter == null) { /* Bluetooth is not supported */ }
      \end{lstlisting}
    \end{exampleblock}
    \begin{small}
    \textit{Note} -- The \codeB{getDefaultAdapter()} method returns the only
    BluetoothAdapter instance shared with the whole system.
    \end{small}
    \begin{exampleblock}{2. Request to enable the BT}
    \begin{lstlisting}[language=Java]
final int REQUEST_ENABLE_BT = 1;

if (!btAdapter.isEnabled()) {
  Intent i = new Intent(BluetoothAdapter.ACTION_REQUEST_ENABLE);
  startActivityForResult(i, REQUEST_ENABLE_BT);
}
    \end{lstlisting}
    \end{exampleblock}
    \begin{itemize}
      \item If the BT is not enabled, through a specific Intent it is possible
      to request the enabling of the BT to the system.
      \begin{itemize}
        \item Enabling must necessarily pass through an explicit authorization
        from the user.
      \end{itemize}
      \item The intent can be executed by using the
      \codeB{startActivityForResult()} method, whose \textit{response} (eventual
      confirmation of enabling) can be intercepted by implementing the
      appropriate callback (\codeB{onActivityResult()}) on the affected
      activity.
    \end{itemize}

    \begin{exampleblock}{Example -- BT Initilization}
      \begin{lstlisting}[language=Java]
public class MyActivity extends Activity{
  private BluetoothAdapter btAdapter;
  private final int ENABLE_BT_REQ = 1;

  @Override 
  protected void onCreate(Bundle savedInstanceState){
    super.onCreate(savedInstanceState);

    /* ... */

    btAdapter = BluetoothAdapter.getDefaultAdapter();
    
    if(btAdapter == null){
      Log.e("MyApp","BT is not available on this device");
      finish();
    }
      \end{lstlisting}
    \end{exampleblock}		
    \begin{exampleblock}{\vspace{-10pt}}
      \begin{lstlisting}[language=Java]
    if (!btAdapter.isEnabled()){
      startActivityForResult(
        new Intent(BluetoothAdapter.ACTION_REQUEST_ENABLE),
        ENABLE_BT_REQ);
    }
  }
  
  @Override
  public void onActivityResult(int reqID, int res, Intent data){
    if(reqID == ENABLE_BT_REQ && res == Activity.RESULT_OK){
      //BT enabled
    }
        
    if(reqID == ENABLE_BT_REQ && res == Activity.RESULT_CANCELED){
      //BT enabling process aborted
    }
  }
}
      \end{lstlisting}
    \end{exampleblock}
  \end{frame}

\subsection{Devices Discovery}

  \begin{frame}[allowframebreaks]{Search for nearby devices}
    \begin{itemize}
      \item The search and identification of devices to which to connect can be
      done in two different ways:
      \begin{itemize}
        \item Connect to unknown devices (\tcc{device discovery})
        \item Connect to previously \tcc{paired} devices
      \end{itemize}
    \end{itemize}

    \begin{block}{Device Discovery}
      \begin{itemize}
        \item It's a Bluetooth frequency scanning procedure that allows you to
        identify any devices present in the range of visibility
        \item All devices that are visible to others can be identified
        \item The identified devices respond by sharing some information (MAC,
        Address, Device Name, \dots)
        \item In order to activate the connection with one of the identified
        devices, the pairing operation must be performed
      \end{itemize}
    \end{block}

    \begin{block}{Pairing}
      \begin{itemize}\itemsep5pt
        \item It's a procedure for pairing two bluetooth devices, performed by
        the user.
        \item It allows you to store on each device all the information
        necessary to activate a connection between the two.
        \item The fact that two devices are paired does not mean that they are
        connected to each other and can exchange data on a dedicated RFCOMM
        channel.
        \item The coupling is a necessary but not sufficient condition to allow
        the transmission of data on the BT channel from one device to another.
      \end{itemize}
    \end{block}
  \end{frame}



  \begin{frame}[fragile]{Paired Devices List}
    \begin{itemize}
      \item The list of previously paired devices can be obtained using the
      \codeB{getBondedDevices()} function that can be run on the
      BluetoothAdapter instance.
      \begin{itemize}
        \item Returns a \codeB{Set} of \codeB{BluetoothDevice} objects 
      \end{itemize}
    \end{itemize}

    \begin{exampleblock}{Example - Paired devices List}
      \begin{lstlisting}[language=Java]
String BT_TARGET_NAME = "mario-Phone";
BluetoothDevice targetDevice = null;
Set<BluetoothDevice> pairedList = btAdapter.getBondedDevices();

if(pairedList.size() > 0){
  for(BluetoothDevice device : pairedList){
    if(device.getName() == BT_TARGET_NAME)
      targetDevice = device;
  }
}
      \end{lstlisting}
    \end{exampleblock}
  \end{frame}

  \begin{frame}[fragile,allowframebreaks]{Discovery Mechanism}
    \begin{block}{Guidelines}
      \begin{itemize}\itemsep10pt
        \item Through the BluetoothAdapter instance it is possible to start and
        stop the discovery procedure of other devices within the range of
        visibility.
        \begin{itemize}
          \item Respectively calling the \codeB{startDiscovery()} and
          \codeB{cancelDiscovery()} methods.
        \end{itemize}
        \item The discovery procedure lasts about 12 seconds and scans the
        Bluetooth frequencies.
        \item In order to receive the information of the identified devices, it
        necessary to register a \tcc{BroadcastReceiver} enabled to intercept each
        \codeB{BluetoothDevice.ACTION\_FOUND} \codeA{IntentFilter}
        \item This is a very energy intensive (battery) procedure.
      \end{itemize}
    \end{block}
    
    \begin{exampleblock}{Example - Discovery of nearby devices}
      \begin{lstlisting}[language=Java]
public class MyActivity extends Activity {
  private BluetoothAdapter btAdapter;
  private Set<BluetoothDevice> nbDevices = null;

  private final BroadcastReceiver br = new BroadcastReceiver() {
    @Override
    public void onReceive(Context context, Intent intent) {
      BluetoothDevice device = null;
      
      if(BluetoothDevice.ACTION_FOUND.equals(intent.getAction())) {
        device = intent.getParcelableExtra(BluetoothDevice.EXTRA_DEVICE);
        nbDevices.add(device);
      }
    }
  };
      \end{lstlisting}
    \end{exampleblock}		
    \begin{exampleblock}{\vspace{-10pt}}
      \begin{lstlisting}[language=Java]
@Override
protected void onCreate(Bundle savedInstanceState) {
  super.onCreate(savedInstanceState);
  setContentView(/* ... */);

  btAdapter = BluetoothAdapter.getDefaultAdapter();

  /* Initialize BT ... */
  
  registerReceiver(br, new IntentFilter(BluetoothDevice.ACTION_FOUND));
}

@Override
public void onStart() {
  super.onStart();
  btAdapter.startDiscovery();
}
      \end{lstlisting}
    \end{exampleblock}		
    \begin{exampleblock}{\vspace{-10pt}}
      \begin{lstlisting}[language=Java]
  @Override
  public void onStop() {
    super.onStop();
    btAdapter.cancelDiscovery();
  }
}
      \end{lstlisting}
    \end{exampleblock}
    \begin{small}
      \begin{block}{Some notes}
        \textit{Note (1).} The BroadcastReceiver must be registered in the system
        (using \codeB{registerReceiver()} specifying the IntentFilter enabled to receive).
        \newline\newline
        \textit{Note (2).} To the onReceive method of the BR, the identified device
        is passed through an Intent, from which it is possible to retrieve the
        instance of the device using the \codeB{getParcelableExtra()} function with the
        specific parameter.
      \end{block}
    \end{small}
  \end{frame}
  
\subsection{RFCOMM and UUID}

  \begin{frame}[allowframebreaks]{Creation of the communication channel}
    % \begin{itemize}
    %   \item It follows the same Client-Server scheme of the communication
    %   channels based on the TCP-IP protocol via socket.
    %   \item A communication channel based on the RFCOMM standard.
    % \end{itemize}
    % \vspace{10pt}
    \begin{block}{Server Side}
      \begin{enumerate}
        \item A Thread (\codeA{MasterThread}) is activated which creates a
        serverSocket that waits for connection requests (blocking call of the
        \codeB{accept()} method).
        \begin{itemize}
          \item The serverSocket can be obtained from the BluetoothAdapter using
          the \codeB{listenUsingRfcommWithServiceRecord()} functionality to which a
          generic name for the channel and the UUID must be passed.
        \end{itemize}
        \item When the serverSocket receives a connection request, it returns
        the specific socket on which the channel was activated.
        \item This socket is then passed to the instance of a bluetooth
        connection manager (\codeA{ConnetionManager}) which can be used both to wait
        for incoming messages (raw data) and to send messages to the client. 
      \end{enumerate}
    \end{block}
    \begin{block}{Client Side}
      \begin{enumerate}
        \item You execute a task whose goal is to connect to the server
        \begin{itemize}
          \item The instance of the socket on which to attempt to connect to the
          server can be obtained using the
          \codeB{createRfcommSocketToServiceRecord()} function to which the UUID
          shared with the server is passed.         
          \item This functionality is provided by any instance of
          \codeB{BluetoothDevice}
          \item The \codeB{BluetoothDevice} associated with the server can be
          taken from the paired list
        \end{itemize}
        \item If the connection is successful, the client-side connection is
        managed by the same \codeA{ConnectionManager} present in the server.
      \end{enumerate}
    \end{block}
  \end{frame}

  \begin{frame}{UUID -- Universal Unique Identifier}
    \begin{block}{Overview}
      \begin{itemize}
        \item Used to uniquely identify the communication channel.
        \item It can be obtained by applying a standard algorithm
        \begin{itemize}
          \item The algorithm convert a UUID string to a specific 128-bit value.
        \end{itemize}
        \item As in Java, you can use the \codeB{fromString(String rep)} utility
        function provided by the \codeB{java.util.UUID} class.
        \begin{itemize}
          \item This representation can be obtained using the \codeB{randomUUID()}
          method provided by the same class.
          \item There are several generators that can be used online (eg
          \texttt{\url{www.uuidgenerator.net}}).
        \end{itemize}
        \item Client and Server must share the same UUID in order to communicate.
      \end{itemize}
    \end{block}
  \end{frame}

  \begin{frame}{UUID for Embedded Devices}
    \begin{block}{Conventional UUID}
      \begin{itemize}\itemsep10pt
        \item For all devices for which it is not possible to make an explicit
        paring (i.e. confirming the shared code on both devices), the UUID to be
        used is established by convention.
        \item This is true for almost all embedded devices, unless their UUID
        has been explicitly changed. 
      \end{itemize}
    \end{block}
    \begin{block}{UUID standard}
      \centering\codeA{00001101-0000-1000-8000-00805F9B34FB}
    \end{block}
  \end{frame}

  \subsection{Implementation Example}
    \begin{frame}{Implementation Example}
      \begin{center}
        \emph{See the AndroidBluetoothEx example provided.}
      \end{center}
    \end{frame}

%\begin{frame}[fragile,allowframebreaks]{Canale di Comunicazione -- Implementazione}
%\begin{exampleblock}{MasterThread (lato server)}
%\begin{lstlisting}[language=Java]
%public class MasterThread extends Thread{
%
%  private BluetoothAdapter btAdapter;
%  private BluetoothServerSocket serverSocket = null;
%
%  private boolean connectionAccepted = false;
%  private boolean stop = false;
%
%  public MasterThread(UUID uuid, String name){
%    btAdapter = BluetoothAdapter.getDefaultAdapter();
%
%    try{
%      serverSocket = btAdapter.listenUsingRfcommWithServiceRecord(name, uuid);
%    } catch (IOException e) {  /* ... */ }
%  }
%\end{lstlisting}
%\end{exampleblock}		
%\begin{exampleblock}{\vspace{-10pt}}
%\begin{lstlisting}[language=Java]
%  public void run(){
%    BluetoothSocket socket = null;
%
%    while(!connectionAccepted && !stop){
%      try{
%        socket = serverSocket.accept();
%      } catch(IOException e){ stop = true; }
%
%      if(socket != null){
%        connectionAccepted = true;
%
%        ConnectionManager cm = ConnectionManager.getInstance();
%        cm.setChannel(socket);
%        cm.start();
%
%        try {
%          serverSocket.close();
%        } catch (IOException e) { /* ... */ }
%\end{lstlisting}
%\end{exampleblock}		
%\begin{exampleblock}{\vspace{-10pt}}
%\begin{lstlisting}[language=Java]
%      }
%    }
%  }
%}
%\end{lstlisting}
%\end{exampleblock}
%
%\begin{exampleblock}{ConnectionTask (lato client)}
%\begin{lstlisting}[language=Java]
%public class ConnectionTask extends AsyncTask<Void, Void, Void>{
%
%  private BluetoothSocket btSocket = null;
%
%  public ConnectionTask(BluetoothDevice server, UUID uuid){
%    try {
%      btSocket = server.createRfcommSocketToServiceRecord(uuid);
%    } catch(IOException e){ /* ... */ }
%  }
%\end{lstlisting}
%\end{exampleblock}		
%\begin{exampleblock}{\vspace{-10pt}}
%\begin{lstlisting}[language=Java]
%  @Override
%  protected Void doInBackground(Void... params){
%    try{ btSocket.connect(); }
%    catch(IOException connectException){
%        try{ btSocket.close(); }
%        catch(IOException closeException){ /* ... */ }
%        return null;
%    }
%
%    ConnectionManager cm = ConnectionManager.getInstance();
%    cm.setChannel(btSocket);
%    cm.start();
%    return null;
%  }
%  
%  @Override
%  protected void onPostExecute(Void par){ /* Connection active */ }
%}
%\end{lstlisting}
%\end{exampleblock}
%
%\begin{exampleblock}{ConnectionManager (lato server e lato client)}
%\begin{lstlisting}[language=Java]
%public class ConnectionManager extends Thread{
%
%  private BluetoothSocket btSocket;
%  private InputStream btInStream;
%  private OutputStream btOutStream;
%
%  private boolean stop;
%
%  private static ConnectionManager instance = null;
%\end{lstlisting}
%\end{exampleblock}		
%\begin{exampleblock}{\vspace{-10pt}}
%\begin{lstlisting}[language=Java]
%  private ConnectionManager(){
%    stop = true;
%  }
%  
%  public static ConnectionManager getInstance(){
%    if(instance == null)
%      instance = new ConnectionManager();
%    return instance;
%  }
%
%  public void setChannel(BluetoothSocket socket){
%    btSocket = socket;
%
%    try {
%      btInStream = socket.getInputStream();
%      btOutStream = socket.getOutputStream();
%    } catch(IOException e) { /*...*/ }
%\end{lstlisting}
%\end{exampleblock}		
%\begin{exampleblock}{\vspace{-10pt}}
%\begin{lstlisting}[language=Java]
%    stop = false;
%  }
%
%  public void run() {
%    byte[] buffer = new byte[1024];
%    int nBytes = 0;
%
%    while(!stop){
%      try {
%        nBytes = btInStream.read(buffer);
%
%        /* Manage here the data stored in buffer  */
%        
%      } catch (IOException e) { stop = true; }
%    }
%  }
%\end{lstlisting}
%\end{exampleblock}		
%\begin{exampleblock}{\vspace{-10pt}}
%\begin{lstlisting}[language=Java]
%  public boolean write(byte[] bytes){
%    if(btOutStream == null) return false;
%
%    try{
%      btOutStream.write(bytes);
%    } catch(IOException e){ return false; }
%
%    return true;
%  }
%
%  public void cancel(){
%    try{
%      btSocket.close();
%    } catch(IOException e){ /*...*/ }
%  }
%}
%\end{lstlisting}
%\end{exampleblock}
%
%\vspace{20pt}
%
%\begin{exampleblock}{Esempio -- Server App}
%\begin{lstlisting}[language=Java]
%public class ServerMainActivity extends Activity{
%
%  private static final String APP_UUID = "c4daba7b-14c1-4cc1-b117-87ff24b9f21d";
%
%  private BluetoothAdapter btAdapter;
%	
%  @Override
%  protected void onCreate(Bundle savedInstanceState){
%    super.onCreate(savedInstanceState);
%	setContentView(/*...*/);
%		
%	btAdapter = BluetoothAdapter.getDefaultAdapter();	
%		
%	/* Initialize BT ... */	
%  }
%\end{lstlisting}
%\end{exampleblock}		
%\begin{exampleblock}{\vspace{-10pt}}
%\begin{lstlisting}[language=Java]	
%  public void onStart(){
%    super.onStart();
%    
%    UUID uuid = UUID.fromString(APP_UUID);
%    MasterThread th = new MasterThread(uuid, "MyBTApp");
%    th.start();
%  }
%  
%  public void onDestroy(){
%    super.onDestroy();
%    ConnectionManager.getInstance().cancel();
%  }
%  
%  private void sendMessage(String msg){
%    ConnectionManager.getInstance().write(msg.getBytes());
%  }
%}
%
%\end{lstlisting}
%\end{exampleblock}
%
%\begin{exampleblock}{Esempio -- Client App}
%\begin{lstlisting}[language=Java]
%public class ClientMainActivity extends Activity {
%
%  private BluetoothAdapter btAdapter;
%  private BluetoothDevice btServer;
%  
%  private static final String BT_SERVER_NAME = "ServerDevice";
%  private static final String APP_UUID = "c4daba7b-14c1-4cc1-b117-87ff24b9f21d";
%
%  @Override
%  protected void onCreate(Bundle savedInstanceState) {
%    super.onCreate(savedInstanceState);
%    setContentView(/*...*/);
%
%    btAdapter = BluetoothAdapter.getDefaultAdapter();
%	
%    /* Initialize BT ... */
%\end{lstlisting}
%\end{exampleblock}		
%\begin{exampleblock}{\vspace{-10pt}}
%\begin{lstlisting}[language=Java]
%    Set<BluetoothDevice> pairedDevices = 
%      btAdapter.getBondedDevices();
%
%    if (pairedDevices.size() > 0)
%      for (BluetoothDevice device : pairedDevices)
%        if (device.getName().equals(BT_SERVER_NAME))
%          btServer = device;
%  }
%
%  @Override
%  public void onStart() {
%    super.onStart();
%
%    UUID uuid = UUID.fromString(APP_UUID);
%    new ConnectionTask(btServer, uuid).execute();
%  }
%\end{lstlisting}
%\end{exampleblock}		
%\begin{exampleblock}{\vspace{-10pt}}
%\begin{lstlisting}[language=Java]
%  @Override
%  public void onDestroy() {
%   super.onDestroy();
%   ConnectionManager.getInstance().cancel();
%  }
%  
%  private void sendMessage(String msg){
%    ConnectionManager.getInstance().write(msg.getBytes());
%  }
%}
%
%\end{lstlisting}
%\end{exampleblock}	
%\end{frame}

%\section{Bluetooth Low Energy}
%
%\begin{frame}{Bluetooth Low Energy (BLE)}
%\begin{itemize}\itemsep10pt
%\item Le versione 4.0 del protocollo Bluetooth, BLE o ``Bluetooth Smart'', è supportata in Android dalla versione 4.3 del sistema operativo (API Level 18).
%\item Tra le altre funzionalità, abilita la scansione dei device BLE in Peripheral Mode.
%\item L'uso del BLE in Android è analogo a quanto detto per lo standard Bluetooth, con alcuni accorgimenti per quanto riguarda la discovery dei dispositivi nelle vicinanze.
%\end{itemize}
%\vfill
%\link{developer.android.com/guide/topics/connectivity/bluetooth-le.html}
%\end{frame}


\section*{References}
%-----------------------
%--- ONLINE RESURCES ---
%-----------------------

\begin{frame}{References - Online Resources}
\begin{thebibliography}{99}
\setbeamertemplate{bibliography item}[online]

\bibitem{adAPIguide} Android Developers - Guide
\newblock \link{https://developer.android.com/guide/}

\bibitem{adAPIreference} Android Developers - API Reference
\newblock \link{https://developer.android.com/reference/}

\bibitem{adTraining} Android Developers - Samples
\newblock \link{https://developer.android.com/samples/}

\bibitem{adTraining} Android Developers - Design \& Quality
\newblock \link{https://developer.android.com/design/}

\end{thebibliography}
\end{frame}

%-----------------------
%--- BOOKS -------------
%-----------------------

%\begin{frame}{Riferimenti - Libri}
%\begin{thebibliography}{99}
%\setbeamertemplate{bibliography item}[book]
%
%\bibitem{Mednieks11} Zigurd Mednieks, Laird Dornin, G. Blake Meike, Masumi Nakamura
%\newblock \emph{Programming Android}
%\newblock O'Reilly, 2011
%
%\bibitem{Haseman13} Chris Haseman, Kevin Grant
%\newblock \emph{Beginning Android Programming: Develop and Design}
%\newblock Peachpit Press, 2013
%
%\bibitem{Schwarz13} Ronan Schwarz, Phil Dutson, James Steele, Nelson To
%\newblock \emph{The Android Developer's Cookbook : Building Applications with the Android SDK}
%\newblock Addison-Wesley, 2013 
%
%\bibitem{Neil14} Theresa Neil
%\newblock \emph{Mobile Design Pattern Gallery: UI Patterns for Smartphone App}
%\newblock O'Relly, Second Edition, 2014
%\end{thebibliography}
%\end{frame}

\end{document}